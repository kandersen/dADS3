\documentclass[a4paper, 12pt]{article}
\usepackage[applemac]{inputenc}
\usepackage[sc]{mathpazo}
\linespread{1.05}         % Palatino needs more leading (space between lines)
\usepackage[T1]{fontenc}
\usepackage[english]{babel}

\usepackage{listings}
\lstloadlanguages{Haskell}
\lstnewenvironment{code}
    {\lstset{}%
      \csname lst@SetFirstLabel\endcsname}
    {\csname lst@SaveFirstLabel\endcsname}
    \lstset{
      basicstyle=\small\ttfamily,
      flexiblecolumns=false,
      basewidth={0.5em,0.45em},
      literate={+}{{$+$}}1 {/}{{$/$}}1 {*}{{$*$}}1 {=}{{$=$}}1
               {>}{{$>$}}1 {<}{{$<$}}1 {\\}{{$\lambda$}}1
               {\\\\}{{\char`\\\char`\\}}1
               {->}{{$\rightarrow$}}2 {>=}{{$\geq$}}2
               {<-}{{$\leftarrow$}}2
               {<=}{{$\leq$}}2 {=>}{{$\Rightarrow$}}2 
               {\ .}{{$\circ$}}2 {\ .\ }{{$\circ$}}2
               {>>}{{>>}}2 {>>=}{{>>=}}2
               {|}{{$\mid$}}1               
    }

\begin{document}
Testing a data structure against a specification is traditionally done
using some permutation of black-box and unit testing, as done in the
previous reports. Implementations in C were tested against
hit-and-miss unit tests at the whim of the programmers, to the extent
that their imagination and endurance permitted.

We can do one better with the levels of abstraction permitted by a
functional language -- in particular the type system of Haskell.

%TODO Insert ref to Hughes & Claessen
We test the implementations by using a concept from operational
semantics known as \emph{observational equivalence}, as detailed by
John Hughes \& Koen Claessen REF. It uses the idea of an
\emph{evaluation context}, a notion of ``the surrounding program'',
and of \emph{observable results}, the effects visible to the
surrounding program caused by running a particular piece of code. Two
pieces of code are then observationally equivalent if they produce the
same observable result in all evaluation contexts.

We here, like the paper, restrict ourselves to ``queue programs'' --
the subset of Haskell dealing only with our queue interface. While we,
for correctness, should consider all of Haskell as evaluation
contexts, we are interested in the behaviour of our implementations,
and -- as opposed to C -- there is no way to manipulate the state of
the queue without going through the interface. Without pointer
arithmetic, side effects and associated suspects, we can be certain of
the absence inadvertant mutation of the international structure of the
queue.

A queue is an inherently ``statefull'' object, and so is typically
used in an imperative fashion. As such we consider imperative-style
monadic programs of Haskell as the context, which only manipulates the
queue through the interface, as argued above. We define a datatype for
representing such contexts:

\begin{lstlisting}
data Action = Inject Int
            | Pop
            | Peak
            | Return (Maybe Int)
\end{lstlisting}

representing an inject, a pop, a peak, and the binding of the result
of a peak, respectively. This last action is needed to argue that
instead of obtaining an element from a queue, if we know that element,
we might as well explcitly write that element in the program
ourselves.

We then define a function ``executing'' an evaluation context on a
given queue, collecting observables as it goes. It reads equationally
as one would expect.

\begin{lstlisting}
perform :: Queue q => q Int -> [Action] -> [Maybe Int]
perform q []     = []
perform q (a:as) =
  case a of
    Inject n -> perform (inject n q) as
    Pop      -> perform (pop q) as
    Peak     -> peak q : perform q as
    Return m -> m : perform q as
\end{lstlisting}

Finally, we define a testable property on queues, that asserts that in
all evaluation contexts (prefixes and suffixes of actions on queues)
we observe the same results from evaluating the queue in that context.

\begin{lstlisting}
equiv :: Queue q => q Int -> [Action] -> [Action] -> Property
equiv q c c' =
  forAll (actions 0) $ \pref ->
  forAll (actions (delta (pref ++ c))) $ \suff ->
  let
    observe x =
      perform q (pref ++ x ++ suff)
  in
   observe c == observe c'
\end{lstlisting}

%TODO REF
Finally, we wire the property into the QuickCheck library (REF HUGHES)
which generates random evaluation contexts, and verifies 5 assertions
on queues, which according to Hughes \& Claeseen (REF) defines a
suitable specification of queues:

\begin{lstlisting}
prop_PeakInject q m n =
  equiv q [Inject m, Inject n, Peak] [Inject m, Peak, Inject n]
prop_InjectPop q m n =
  equiv q [Inject m, Inject n, Pop] [Inject m, Pop, Inject n]
prop_PeakEmpty q =
  equivFromEmpty q [Peak] [Return Nothing]

prop_PeakInjectEmpty q m =
  equivFromEmpty q [Inject m, Peak] [Inject m, Return (Just m)]
prop_InjectPopEmpty q m =
  equivFromEmpty q [Inject m, Pop] []
\end{lstlisting}

The essence of the argument for the specification is to ``bubble''
peeks and pops towards the beginning of the program with the first 3
properties, and then eliminate them with the 2 following properties,
thus ``normalizing'' programs on queues.

Finally we run the test for each of our 4 implementations, creating a
hundred programs of an average length of 8 for each property for a
total of $4 \cdot 5 \cdot 100 = 2000$ unit tests. These are just
default settings and as good as any, but more tests naturally
strengthen the claim that the implementations meet the specifications.

\begin{lstlisting}
runTests :: IO ()
runTests = do test "SimpleQueue" (empty :: BasicQueue Int)
              test "SmartListQueue" (empty :: PairQueue Int)
              test "OkasakiQueue" (empty :: OkasakiQueue Int)
              test "RealTimeQueue" (empty :: RealTimeQueue Int)
  where
    test s q = do putStrLn $ "### Testing " ++ s ++ " ###"
                  suite q
\end{lstlisting}

It is unfortunate that the tests do not conclusively prove the
correctness of our implementations. If only there was a way...

It turns out, there is! As all our data structures are purely
functional and do not use general recursion in the operations, we can
implement the data structures in the Proof Assistant Coq. Which we
did. Coq allows us to state properties on functional programs,
\emph{conclusively prove them}, and then extract the functional
programs to e.g. Haskell. As an added benefit, just by virtue of using
Coq as an implementation we certify our functions total and
terminating, as Coq does not accept partially defined functions.

We start by giving an abstract module declaration of a queue
implementaion.

\begin{lstlisting}
Module Type QUEUE.
  Parameter Q       :           Type -> Type. 
  Parameter inv     : forall A, Q A -> Prop.
  Parameter empty   : forall A, Q A.
  Parameter inject  : forall A, A -> Q A -> Q A.
  Parameter pop     : forall A, Q A -> Q A.
  Parameter peak    : forall A, Q A -> option A.  
  Parameter to_list : forall A, Q A -> list A.
\end{lstlisting}

The \lstinline{inv} field is a predicate on a queue representing
whatever invariants might be applicable for a given implementation,
and the \lstinline{to_list} function is used to convert a queue to an
ordered list of elements, which we shall use as our model for
specifying queues. Using an abstract model (ordered lists) and then
showing our expectations in the model are satisfied by the
implementation is much in the spirit of observational equivalence: if
looks like we expect in all ways we can measure it, it must be the same.

Next, we define a series of proof obligations, to describe what must
be proven about a given implementation. Among them are that the
invariant is satisfied by an empty queue, and preserved by all
operations. Do not be alarmed by the use of ``Axiom'': it means ``to
be proven'', not ``to be assumed''.

\begin{lstlisting}
  Definition represents {A : Type} (q : Q A) (l : list A) : Prop :=
    to_list _ q = l.

  Axiom empty_inv :
    forall A, 
      inv A (empty A).

  Axiom inject_inv :
    forall A q x,
      inv A q ->
      inv _ (inject _ x q).

  Axiom pop_inv :
    forall A q,
      inv A q ->
      inv _ (pop _ q).
\end{lstlisting}

The \lstinline{represents} field is used to tie a given instance of a
queue to its representation in the model. With that we can define the
following proof obligations:

\begin{lstlisting}
  Axiom empty_is_empty : forall A,
    represents (empty A) nil.
\end{lstlisting}

The above dictates an empty queue represents the empty list of elements.

\begin{lstlisting}  
  Axiom inject_spec :
    forall A l q x,
      represents q l ->
      inv _ q ->
      represents (inject A x q) (l ++ (x :: nil)).
\end{lstlisting}

The above states that injecting an element corresponds to postpending
an element to the ordered list.

\begin{lstlisting}  
  Axiom pop_empty :
    forall A q,
      represents q nil ->
      inv _ q ->
      represents (pop A q) nil.
\end{lstlisting}

The above states that that popping the empty queue return the empty queue.

\begin{lstlisting}  
  Axiom pop_xs :
    forall A q x xs,
      represents q (x::xs) ->
      inv _ q ->
      represents (pop A q) xs.
\end{lstlisting}

The above states that popping an element from a queue which represents
the list starting with \lstinline{x} followed by \lstinline{xs} yields
a queue representing \lstinline{xs}.

\begin{lstlisting}    
  Axiom peak_empty_spec :
    forall A q,
      represents q nil ->
      inv _ q -> 
      peak A q = None.
\end{lstlisting}

The above states that peeking an element from an empty queue yields nothing.

\begin{lstlisting}  
  Axiom peak_spec :
    forall A q x xs,
      represents q (x::xs) ->
      inv _ q ->
      peak A q = Some x.

End QUEUE.
\end{lstlisting}

Finally, we state that peeking from a queue representing the ordered
sequence starting with \lstinline{x} yields that \lstinline{x}.

The basic, naive functional queue is an implementation of our model,
and so the proofs are largely uninteresting by construction. So are
the naive implementation based on a pair of queues. We wont give a
detailed account of the proofs of each implementation, as the proof
scripts are quite monstrous to present in writing. We refer curious
readers to \lstinline{Queue.v} in the accompanying code.

Finally, when all proof obligations are satisfied we run
e.g.
\begin{lstlisting}
Extraction "PairQueue.hs" PairQueue
\end{lstlisting}
to generate a Haskell file with the definitions given in the
module. While slightly unidiomatic in style, the generated code is
exactly as we would have handwritten, and so suffer no performance
penalties.

The generated files are exactly those used in our benchmarks and the
previously outlined random testing, so it is no surprise it satisfied
the previous specification!

We might remark that Coq is a strict programming language which we use
to reason about programs to be run in a lazy setting. This is
inherently not a problem as none of the implementations exploit lazy
capabilities of lazy evaluation but use it only to argue performance
characteristics. We have just proven they have the desired semantics
and so will behave the same in a lazy setting, with the added benefit
of ``lazy performance''.

\end{document}
