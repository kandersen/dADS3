\documentclass[a4paper, 12pt]{article}
\usepackage[T1]{fontenc}
\usepackage[english]{babel}
\usepackage[applemac]{inputenc}

\usepackage[sc]{mathpazo}
\linespread{1.05}

\title{Red-Black Trees}
\date{}
\author{Kristoffer}
\begin{document}
\maketitle
\section{Intro}
The simplest search tree we know is the binary search tree (BST). Its
invariant is simple, and both querying and updating is formulated
easily, both iteratively and recursively. The runtime guarantees are,
while equally easy to argue, not optimal.

The root of the problem is an issue of balancing. A plain BST
maintains no notion of balance, so a pathological use can create 'flat
trees' where all nodes end up along a single spine, giving worst-case
linear time for most operations.

One solutions is a red-black tree (RBT), where the idea is to maintain
additional structure in order to remain balanced.

The remainder of this section describes the structure of an RBT, and
describes the operations which we have implemented with the aim of
empirical comparison with van Emda Boas trees.

\section{Red-Black Trees}\label{sec:rbt}

A red black tree is essentially an ordinary BST with an additional bit
of information per node: the ``color'' of that node, black or
red. Through the use of a handful of invariants, the color information
is used to balance the RBT. So in addition to the usual fields of key,
parent pointer, left and right child pointers, we also store a color
bit.

The fact that an RBT is just an augmented BST means that the basic
queries essentially need no changes to work on RBTs, and, indeed,
Cormen et al. (REF) on which we base our implementation, defers to the
section on simple BSTs.

The invariants that we place on an RBT, in addition to the basic
search tree property, is as follows:

\begin{enumerate}
  \item The root is black
  \item All leaves are black
  \item If a node is red, both its children are black
  \item For each node, all paths from node to descendant leaves
    contain the same number of black nodes.
\end{enumerate}

See Lemma 13.1 in Cormen et al (REF) for how these last two properties
ensure balancing.

\section{Operations}

\subsection{Queries}

No queries exploit the RBT structure, and so work procedurally like the
operations known from simple BSTs. The running times, however, benefit
from the balancing guarantees of RBTs.

\paragraph{Minimum} The minimum node of a search tree is its left-most
node, if any. This is given by the basic search tree invariant. By
virtue of being balanced, the minimum node lies at a maximum depth of
$\log(n)$, where $n$ is the number of nodes in the tree, and the
worst-case running time is thus $O(\log(n))$.

\paragraph{Maximum} Symmetric to Minimum.

\paragraph{Predecessor} The Predecessor query asks, given a particular
node in a search tree: ``what is the greatest node smaller than this
particular node?'' In the case of an internal node, the answer can be
constructed using Maximum, as we simply query the left subtree of the
given node for its maximum node: running time proportional to the
depth of the given node. In the case of a leaf, we must find the first
ancestor such that the given node is in its right sub-tree. This will
ensure $Successor \circ Predecessor = id$. The running time is again
proportional to the depth of the given node, and so is always
$O(\log(n))$.

\paragraph{Successor} Symmetric to Predecessor.

\paragraph{Search} The Search query asks, given a key and a search
tree, ``does the tree contain a node with the given key?'', and relies
entirely on the basic search tree invariant. If the node at the root
of the given tree has the desired key, we have found what we are
looking for. Otherwise, the desired key is either smaller or larger
than the key at the root of the given tree, and we can thus
recursively search either the left or right tree, respectively. We
answer negatively upon hitting a leaf. The running time is
proportional to the depth of the initial tree, and thus runs
$O(\log(n))$ where $n$ is the number of nodes in the tree.

\subsection{Updates}

Updating operations on RBTs are complicated by the need to maintain
the RBT invariants described in section \ref{sec:rbt}.

\paragraph{Insert} To maintain the search-tree invariant, the
RBT insertion procedure works operationally the same as the insertion
procedure, and then does some additional work to reestablish the RBT
invariants. The first phase of searching runs proportional to the
depth of the tree, and the reestablishing runs back up the tree in the
worst case, performing a constant amount of work at each node. This
gives a grand total of $O(\log(n))$ work for insertions.

\paragraph{Delete} Deletion is similar in argument to insertion,
albeit complicated when the node to be removed is black. It does a
search down, and a fix up, for a run-time of $O(\log(n))$.

\end{document}
