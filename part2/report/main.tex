\documentclass[oneside,11pt,openright]{report}

\usepackage[latin1]{inputenc}
\usepackage[american]{babel}
\usepackage{a4}
\usepackage{latexsym}
\usepackage{amssymb}
\usepackage{amsmath}
\usepackage{epsfig}
\usepackage[T1]{fontenc}
\usepackage{lmodern}
\usepackage[labeled]{multibib}
\usepackage{color}
\usepackage{datetime}
\usepackage{epstopdf} 
\usepackage{graphicx}
\usepackage{pgfplots}
\usepackage[format=hang]{caption}

\usepackage{pgf,tikz}
\usepackage{comment}
\usetikzlibrary{arrows,automata}
\usetikzlibrary{backgrounds,fit}
\usetikzlibrary{shapes,patterns}
\usetikzlibrary{calc,chains,positioning}

\renewcommand*\ttdefault{txtt}
\newcommand{\BigO}[1]{\ensuremath{\operatorname{O}\left(#1\right)}}
\newcommand{\BigT}[1]{\ensuremath{\Theta\left(#1\right)}}
\newcommand{\specialcell}[2][c]{%
  \begin{tabular}[#1]{@{}c@{}}#2\end{tabular}}
% see http://imf.au.dk/system/latex/bog/

\newcommand{\adjustimg}{% Horizontal adjustment of image
  \ifodd\value{page}\hspace*{\dimexpr\evensidemargin-\oddsidemargin}\else\hspace*{-\dimexpr\evensidemargin-\oddsidemargin}\fi%
}
\newcommand{\centerimg}[2][width=\textwidth]{% Center an image
  \makebox[\textwidth]{\adjustimg\includegraphics[#1]{#2}}%
}
\newcommand{\MakeHeap}{\textsc{MakeHeap}}
\newcommand{\FindMin}{\textsc{FindMin}}
\newcommand{\Insert}{\textsc{Insert}}
\newcommand{\DeleteMin}{\textsc{DeleteMin}}
\newcommand{\DecreaseKey}{\textsc{DecreaseKey}}
\newcommand{\Delete}{\textsc{Delete}}
\newcommand{\Meld}{\textsc{Meld}}
\newcommand{\Dijkstra}{\textsc{Dijkstra}}
\newcommand{\NULL}{\textbf{null}}

\newcommand{\Member}{\textsc{Member}}
\newcommand{\Minimum}{\textsc{Minimum}}
\newcommand{\Maximum}{\textsc{Maximum}}
\newcommand{\Predecessor}{\textsc{Predecessor}}
\newcommand{\Successor}{\textsc{Successor}}

\newcommand{\HIGH}{\textsf{high}}
\newcommand{\LOW}{\textsf{low}}
\newcommand{\INDEX}{\textsf{index}}
\newcommand{\HIGHER}{\sqrt[\uparrow]{u}}
\newcommand{\LOWER}{\sqrt[\downarrow]{u}}

\begin{document}

%%%%%%%%%%%%%%%%%%%%%%%%%%%%%%%%%%%%%%%%%%%%%%%%%%%%%%%%%%%%%%%%%%%%%%%

\pagestyle{empty} 
\pagenumbering{roman} 
\vspace*{\fill}\noindent{\rule{\linewidth}{1mm}\\[4ex]
{\Huge\sf Binary Heaps, Fibonacci Heaps and\\[2ex]Dijkstras shortest path}\\[4ex]
{\huge\sf Kristoffer Just Andersen, 20051234\\[2ex]
\huge\sf Troels Leth Jensen, 20051234 \\[2ex]
\huge\sf Morten Krogh-Jespersen, 20022362}\\[2ex]
\noindent\rule{\linewidth}{1mm}\\[4ex]
\noindent{\Large\sf Project 1, Advanced Data Structures 2013, Computer Science\\[1ex] 
\monthname\ \the\year  \\[1ex] Advisor: Gerth St�lting Brodal\\[15ex]}\\[\fill]}
\epsfig{file=logo.eps}\clearpage

%%%%%%%%%%%%%%%%%%%%%%%%%%%%%%%%%%%%%%%%%%%%%%%%%%%%%%%%%%%%%%%%%%%%%%%

\tableofcontents
\pagenumbering{arabic}
\setcounter{secnumdepth}{2}

%%%%%%%%%%%%%%%%%%%%%%%%%%%%%%%%%%%%%%%%%%%%%%%%%%%%%%%%%%%%%%%%%%%%%%%

\chapter{Introduction}

needs content

\chapter{Bit-vector}

A bit-vector or a bit array is a datastructure that supports
$\Insert$, $\Delete$ and $\Member$ in constant
time~\cite[p. 532]{ITA09}. The bit-vector consists of a set of bits,
for which the $i'th$ bit indicates the presence of the
$i'th$ key. Thus, in order to support queries over an universe of size $u$ we must
have a bit-vector with $u$ bits.

To $\Insert$ some key $i$ into the bit-vector the $i'th$ bit is set to
true, similarly to $\Delete$ the same key $i$, the $i'th$ bit is
cleared. The $\Member$ test for key $i$ is done by checking the $i'th$
bit. All these operations are supported easily through bit
shifts. 

Please note, that these bit operations assume that the universe size
does not exceed the word size. If that is not the case, extra
computation is needed to first find the correct block; this however is
also constant time.

The bit-vector performs badly for $\Minimum$, $\Maximum$,
$\Predecessor$ and $\Successor$ because the entire array could be
traversed in search for the answer.

\begin{center}
  \begin{tabular}{ l | c | c}
    Operation & Bit-Vector & Bit-Vector w. min \& max \\ \hline
    \Member & $\BigT{1}$ & $\BigT{1}$ \\ 
    \Minimum & $\BigO{u}$ & $\BigT{1}$ \\ 
    \Maximum & $\BigO{u}$ & $\BigT{1}$ \\ 
    \Predecessor & $\BigO{u}$ & $\BigO{u}$ \\ 
    \Successor & $\BigO{u}$ & $\BigO{u}$ \\ 
    \Insert & $\BigT{1}$ & $\BigT{1}$ \\ 
    \Delete & $\BigT{1}$ & $\BigO{u}$ \\
  \end{tabular}
\end{center}

Alternatively, $\Minimum$ and $\Maximum$ can be maintained during all
changes to the bit vector, i.e. during $\Insert$ and $\Delete$. With this
approach the running time of $\Delete$ would in worst case be completely
dominated by the linear search for a new $\Minimum$ or $\Maximum$ while
querying for the $\Minimum$ or $\Maximum$ would become constant time
operations. For the remaining part of this report, we will refer to the Bit-Vector as the version that maintains $\Minimum$ and $\Maximum$.

\chapter{Red-Black Tree}

The simplest search tree we know is the binary search tree (BST). Its
invariant is simple, and both querying and updating is formulated
easily, both iteratively and recursively. The runtime guarantees are,
while equally easy to argue, not optimal.

The root of the problem is an issue of balancing. A plain BST
maintains no notion of balance, so a pathological use can create 'flat
trees' where all nodes end up along a single spine, giving worst-case
linear time for most operations.

One solutions is a red-black tree (RBT), where the idea is to maintain
additional structure in order to remain balanced.

The remainder of this section describes the structure of an RBT, and
describes the operations which we have implemented, that has the following time complexities with compared with the Bit-Vector:

\begin{center}
  \begin{tabular}{ l | c | c}
    Operation & Bit-Vector & Red-Black tree \\ \hline
    \Member & $\BigT{1}$ & $\BigO{\log n}$ \\ 
    \Minimum & $\BigT{1}$ & $\BigO{\log n}$ \\ 
    \Maximum & $\BigT{1}$ & $\BigO{\log n}$ \\ 
    \Predecessor & $\BigO{u}$ & $\BigO{\log n}$ \\ 
    \Successor & $\BigO{u}$ & $\BigO{\log n}$ \\ 
    \Insert & $\BigT{1}$ & $\BigO{\log n}$ \\ 
    \Delete & $\BigO{u}$ & $\BigO{\log n}$ \\
  \end{tabular}
\end{center}


\section{Red-Black Trees}\label{sec:rbt}

A red black tree is essentially an ordinary BST with an additional bit
of information per node: the ``color'' of that node, black or
red. Through the use of a handful of invariants, the color information
is used to balance the RBT. So in addition to the usual fields of key,
parent pointer, left and right child pointers, we also store a color
bit.

The fact that an RBT is just an augmented BST means that the basic
queries essentially need no changes to work on RBTs, and, indeed,
Cormen et al. (REF) on which we base our implementation, defers to the
section on simple BSTs.

The invariants that we place on an RBT, in addition to the basic
search tree property, is as follows:

\begin{enumerate}
  \item The root is black
  \item All leaves are black
  \item If a node is red, both its children are black
  \item For each node, all paths from node to descendant leaves
    contain the same number of black nodes.
\end{enumerate}

See Lemma 13.1 in Cormen et al (REF) for how these last two properties
ensure balancing.

\section{Operations}

\subsection{Queries}

No queries exploit the RBT structure, and so work procedurally like the
operations known from simple BSTs. The running times, however, benefit
from the balancing guarantees of RBTs.

\paragraph{Minimum} The minimum node of a search tree is its left-most
node, if any. This is given by the basic search tree invariant. By
virtue of being balanced, the minimum node lies at a maximum depth of
$\log(n)$, where $n$ is the number of nodes in the tree, and the
worst-case running time is thus $O(\log(n))$.

\paragraph{Maximum} Symmetric to Minimum.

\paragraph{Predecessor} The Predecessor query asks, given a particular
node in a search tree: ``what is the greatest node smaller than this
particular node?'' In the case of an internal node, the answer can be
constructed using Maximum, as we simply query the left subtree of the
given node for its maximum node: running time proportional to the
depth of the given node. In the case of a leaf, we must find the first
ancestor such that the given node is in its right sub-tree. This will
ensure $Successor \circ Predecessor = id$. The running time is again
proportional to the depth of the given node, and so is always
$O(\log(n))$.

\paragraph{Successor} Symmetric to Predecessor.

\paragraph{Search} The Search query asks, given a key and a search
tree, ``does the tree contain a node with the given key?'', and relies
entirely on the basic search tree invariant. If the node at the root
of the given tree has the desired key, we have found what we are
looking for. Otherwise, the desired key is either smaller or larger
than the key at the root of the given tree, and we can thus
recursively search either the left or right tree, respectively. We
answer negatively upon hitting a leaf. The running time is
proportional to the depth of the initial tree, and thus runs
$O(\log(n))$ where $n$ is the number of nodes in the tree.

\subsection{Updates}

Updating operations on RBTs are complicated by the need to maintain
the RBT invariants described in section \ref{sec:rbt}.

\paragraph{Insert} To maintain the search-tree invariant, the
RBT insertion procedure works operationally the same as the insertion
procedure, and then does some additional work to reestablish the RBT
invariants. The first phase of searching runs proportional to the
depth of the tree, and the reestablishing runs back up the tree in the
worst case, performing a constant amount of work at each node. This
gives a grand total of $O(\log(n))$ work for insertions.

\paragraph{Delete} Deletion is similar in argument to insertion,
albeit complicated when the node to be removed is black. It does a
search down, and a fix up, for a run-time of $O(\log(n))$.


\chapter{Van Emde Boas Tree}

A van Emde Boas tree is a recursive data structure that supports
finding the predecessor and successor in \BigO{\log \log m} time where
$m$ is the size of the universe, in other words, the amount of
distinct keys with a total ordering that the tree
support~\cite[p. 545]{ITA09}~\cite{VEB04}. Because van Emde Boas Trees
allow universe sizes of any power of 2, we denote $\HIGHER =
2^{\lceil{(\log u)/2}\rceil}$ and $\LOWER = 2^{\lfloor{(\log
    u)/2}\rfloor}$.  Each level of the van Emde Boas Tree has a
universe $u$ and it contains $\HIGHER$ clusters/bottom van Emde Boas
trees of universe size $\LOWER$ and one auxilary van Emde Boas Tree of
size $\HIGHER$ we denote top.  For each recursion the universe shrinks
by $\sqrt{u}$.

For each tree we have two attributes that store the minimum and the
maximum key. The minimum key cannot be found in any of the bottom
trees, neither can the maximum unless it differs from the minimum,
which happens if there are more than one element in the tree. The
attributes helps reduce the number of recursive calls, because one can
in constant time decide if a value lies within the range, without the
need to recurse.

Van Emde Boas Trees utilizes that keys are represented as bits and if
we view a key $x$ as a $\log u$-bit binary integer, we can divide the
bits up in a most significant and a least significant part. The most
significant part of the bit vector identifies the cluster
$\lfloor{x/\sqrt{u}}\rfloor$ where $x$ will apear in position $x \mod
\sqrt{u}$ in the beforementioned bottom tree. Let us denote $\HIGH(x)
= \lfloor{x/\sqrt{u}}\rfloor$ and $\LOW(x) = x \mod \sqrt{u}$ then we
get the identity $x = \HIGH(x) \sqrt{u} + \LOW(x)$.

Below is the time-complexities of each operation listed for the
corresponding data structure and let $u$ denote the size of the
universe and $n$ the number of keys stored in the structure:

\begin{center}
  \begin{tabular}{ l | c | c | c }
    Operation & Bit-Vector & Red-Black Tree & van Emde Boas Tree \\ \hline
    \Member & $\BigT{1}$ & $\BigO{\log n}$ & $\BigO{\log \log u}$ \\ 
    \Minimum & $\BigT{1}$ & $\BigO{\log n}$ & $\BigO{1}$\\ 
    \Maximum & $\BigT{1}$ & $\BigO{\log n}$ & $\BigO{1}$ \\ 
    \Predecessor & $\BigO{u}$ & $\BigO{\log n}$ & $\BigO{\log \log u}$  \\ 
    \Successor & $\BigO{u}$ & $\BigO{\log n}$ & $\BigO{\log \log u}$ \\ 
    \Insert & $\BigT{1}$ & $\BigO{\log n}$ & $\BigO{\log \log u}$ \\ 
    \Delete & $\BigO{u}$ & $\BigO{\log n}$ & $\BigO{\log \log u}$ \\
  \end{tabular}
\end{center}

\section{Finding minimum or maximum key}

This is a constant operation, since the van Emde Boas Tree directly stores the minimum and maximum element.

\section{Finding a member}

Finding out if a key is stored in a van Emde Boas Tree is simple. Either the key is the minimum or the maximum element or else recurse until we find the element. It is easy to figure out which tree to recurse on, since this is the $\HIGH(x)$-th tree, and we just have to search in the smaller tree for $\LOW(x)$.

So how long does the search take in worst-case? The data structure can be described by the following recurrence function: 
\begin{align*}
    T(u) \leq T(\HIGHER) + \BigO{1}
\end{align*}
If we let $m = \log u$ and realize that $\lceil{m/2}\rceil \leq 2m/3$ for $m \geq 2$, which is the leaf size of the van Emde Boas Tree, we get:

\begin{align*}
    T(u)  & \leq T(\HIGHER) + \BigO{1} \\
    \Downarrow \\
    T(2^m) & \leq T(2^{\lceil{m/2}\rceil}) + \BigO{1} \\
    \Downarrow \\
    T(2^m) & \leq T(2^{2m/3}) + \BigO{1} \\
    \Downarrow \\
    S(m) & \leq S(2m/3) + \BigO{1} \\
\end{align*}

By the master thorem~\cite[p. 93]{ITA09}, has the solution $S(m) = \BigO{\log m}$. Because $T(u) = T(2^m)$ we get $\BigO{\log m} = \BigO{\log \log u}$. Therefore the procedure takes $\BigO{\log \log u}$.

\section{Finding a successor or a predecessor}

As with finding a member, for the base case a successor or predecessor can be found in constant time. If not, we have to determine where to find the next element we are searching for. Finding successor and predecessor is completely analog thus we only describe finding the successor for an element $x$.

First, we check if the sucessor is in the bottom tree at index $\HIGH(x)$. This check ban be done in constant time since it is a matter of checking maximum for that tree. If maximum exist and it is higher than $\LOW(x)$, we know we have to search inside that particular tree. This will take $\BigO{\log\log u}$ time.

If maximum does not exist or it is less than $\LOW(x)$ we use the top tree to search for the successor to $\HIGH(x)$. If we can find such an element it will gives us an index to a bottom tree $b$. Hereafter it is a constant lookup to find $\Minimum(b)$. This case also runs in $\BigO{\log\log u}$ time since we have on search in top and a constant lookup. Therefore, the total running time of successor and predecessor is $\BigO{\log\log u}$.

\section{Insert}

Insert is pretty simple. One of the following can happen:

\begin{itemize}
\item The list is empty, which can be discovered in constant time by checking the minimum attribute. If that is the case, insert just set min and max.
\item If not in the base case, find out if the bottom of $\HIGH(x)$ is empty. If that is the case, we can just set min and max as above and update the top structure. If not, we just call insert on the bottom tree for $\LOW(x)$.
\end{itemize}

Always remember to set max or swap with min (because min was not in the tree, but now has to be since it would no longer be min). Either, the insert function recurses on the top tree of size $\HIGHER$ or on one of the bottom trees of size $\LOWER$ but not on both. Therefore, the running time is at most $\BigO{\log \log u}$.

\section{Delete}

If there is only one element or universe size is 2 is easy to perform delete in constant time. Otherwise, some more work has to be done. Consider what should be done if we try to delete the minimum element. Since min is not stored in the tree, we have to find the minimum element in the bottom trees and make sure we delete it so that it can be placed as the new min. Of course, it can happen that the the bottom tree becomes empty so the top structure has to be updated. It might also be that we remove max but updating max is a constant operation.

In the description above we actually could make two recursive calls; one to update the bottom tree and possibly one to update the top. But if we update the top then the bottom tree would only contain one element. But if that is the case the first recursive function takes constant time. Therefore delete runs in $\BigO{\log \log n}$.

\chapter{Conlusion}

needs contents


%%%%%%%%%%%%%%%%%%%%%%%%%%%%%%%%%%%%%%%%%%%%%%%%%%%%%%%%%%%%%%%%%%%%%%%

\addcontentsline{toc}{chapter}{Bibliography}
\bibliographystyle{plain} 
\bibliography{report}

\end{document}

